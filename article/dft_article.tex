\documentclass[12pt]{article}

\usepackage{sbc-template}

\usepackage{graphicx,url}
\usepackage{mathptmx}
\usepackage{graphicx}
\usepackage{times}
\usepackage{enumitem}
\usepackage[]{algorithm2e}
\usepackage[table,xcdraw]{xcolor}
\usepackage{pdfpages}
\usepackage{booktabs}

%\usepackage[brazil]{babel}   
\usepackage[latin1]{inputenc} 


\title{Discrete Fourier Transform for Dummies}

\author{Alex Frasson and Tiago Augusto Engel\inst{1}}


\address{Universidade Federal de Santa Maria (UFSM)\\ 
	\email{\{afrasson,tengel\}@inf.ufsm.br}
}

\begin{document} 
\maketitle

\section{Introduction}

\section{The Real DFT}

nesse link explica 

http://www.analog.com/media/en/technical-documentation/dsp-book/dsp\_book\_Ch31.pdf

\begin{equation}
ReX(k)=\frac{2}{N} \sum_{n=0}^{N-1}f(n) \cos(2\pi kn/N)
\end{equation}
\begin{equation}
ImX(k)=\frac{-2}{N} \sum_{n=0}^{N-1}f(n) \sin(2\pi kn/N)
\end{equation}

The N sample time domain signal $f(n)$ is decomposed into a set of N/2+1 cosine, and N/2+1 sine waves, with frequencies given by the index $k$. The amplitudes of the cosine waves are contained in $ReX(k)$, while the amplitudes of the sine waves are contained in $ImX(n)$. These  equations operate by correlating the respective cosine or sine wave with the time domain signal.  In spite of using the names: real part and imaginary part, there are no complex  numbers in these equations\cite{Analog2016}.


\section{1D DFT Definition}
\begin{equation}
F(u)=\sum_{x=0}^{M-1}f(x) e^{-j 2\pi u x / N}
\end{equation}
where $u = 0,1,2,\dots, M-1$

\begin{equation}
f(x)=\frac{1}{M}\sum_{u=0}^{M-1}F(u) e^{j 2\pi u x / N}
\end{equation}
where $x = 0,1,2,\dots, M-1$

\subsection{Implementation}

\section{2D DFT Definition}


\begin{equation}
F(k,l)=\frac{1}{MN}\sum_{m=0}^{M-1}\sum_{n=0}^{N-1} f(m,n) e^{-j 2\pi (\frac{km}{M} + \frac{ln}{N})}
\end{equation}
where $m = 0,1,2,\dots, M-1$ and $n = 0,1,2,\dots, N-1$

\begin{equation}
f(m,n)=\sum_{k=0}^{M-1}\sum_{l=0}^{N-1} F(k,l) e^{j 2\pi (\frac{km}{M} + \frac{ln}{N})}
\end{equation}
where $k = 0,1,2,\dots, M-1$ and $l = 0,1,2,\dots, N-1$

\subsection{The spectrum}
\subsection{The magnitude}


\bibliographystyle{sbc}
\bibliography{references}

\end{document}
